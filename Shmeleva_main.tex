\documentclass[14pt, a4paper]{extarticle}
\usepackage{tempora} %Times New Roman alike
\usepackage{cmap}		
\usepackage{mathtext} 	
\usepackage[T2A]{fontenc}
\usepackage[utf8]{inputenc}
\usepackage[english, russian]{babel}	
\usepackage{amsmath,amsfonts,amssymb,mathtools}

\usepackage{icomma} 
\usepackage{pdfpages}
\usepackage{etoolbox} 
\usepackage{pgfplots}
\usepackage{ mathrsfs }
\usepackage{upgreek}
\usepackage{arcs}
\usepackage{subfig}
\usepackage{extsizes} 
\usepackage{geometry} 
	\geometry{top=12 mm}
	\geometry{bottom=32mm}
	\geometry{left=25mm}
	\geometry{right=25mm}
	\pgfplotsset{compat=1.9}
\usepackage{fancyhdr}
\usepackage[argument]{graphicx}
%\graphicspath{{pictures/}}
\DeclareGraphicsExtensions{.pdf,.png,.jpg}
\usepackage{amssymb}
\usepackage{gensymb}
\usepackage{xcolor}
\usepackage{hyperref}
\usepackage[unicode, pdftex]{hyperref}
%\newcommand{\tl}[1]{\textbf{#1}}

%% Настройки колонтитулов
\usepackage{fancyhdr}
\pagestyle{fancy}
\fancyhead{}
\fancyhead[L]{Шмелева Анна}
\fancyhfoffset[L]{0pt}
\fancyhfoffset[R]{0pt}
\headheight=40pt
\fancyhead[C]{\textbf{НИУ ВШЭ}}
\fancyhead[R]{\today}
\fancyfoot{}

\usepackage{lastpage}
\fancyfoot[C]{\thepage}
\fancypagestyle{plain}{\fancyhead{}\renewcommand{\headrule}{}}
\renewcommand{\headrulewidth}{1pt}
\renewcommand{\footrulewidth}{1pt} 
\fancypagestyle{firststyle} %Первая страница
{
\renewcommand{\headrulewidth}{1pt} 
\renewcommand{\footrulewidth}{1pt} 
}

\usepackage{hyperref}
\hypersetup{
    colorlinks=true,
    linkcolor=blue,
    filecolor=magenta,      
    urlcolor=blue,
}

\usepackage{amscd}
\usepackage{dirtytalk}
\usepackage{setspace}

\usepackage{csquotes}% Recommended
\usepackage{biblatex}
\addbibresource{sample.bib}

\begin{document}

\newenvironment{bottompar}{\par\vspace*{\fill}}

\begin{titlepage}
\center % Center everything on the page

\textsc{\LARGE Высшая Школа Экономики\\[0.1cm]\large (Национальный Исследовательский Университет)}%\\[1.5cm] % Name of your university/college
\begin{center}
    \includegraphics[width=0.2\textwidth]{images/hse logo.jpg}
\end{center}
\textsc{\Large Научно-исследовательский семинар}

\vspace{2pt}% Major heading such as course name

\textsc{\large Обзор литературы}\\[0.5cm] % Minor heading such as course title

  \vspace{10pt}
  
    \hrule
    
  \vspace{13pt}
{ \huge \bfseries Классификация криптовалют\\[0.2cm] }

  \vspace{10pt}
    \hrule
  \vspace{15pt}


\begin{minipage}{0.4\textwidth}
  \begin{flushleft} \large
    \textsf{Студент}
    
    Шмелева Анна  \textsc{} \\[-0.15cm]
    БЭАД223
  \end{flushleft}
\end{minipage}
~
\begin{minipage}{0.4\textwidth}
  \begin{flushright} \large
    \textsf{Преподаватель}
    
  Степанченко Д.С.\\[-0.15cm]
    \textsc{} % Supervisor's Name
  \end{flushright}
\end{minipage}

\begin{flushleft}
    \vspace{10cm}
    \textbf{Ключевые слова:} криптовалюта, классификация, протоколы 
\vspace{5mm}

    \textbf{Аннотация:} В статье рассмеотрены классификации криптовалют по различным признакам 
\end{flushleft}


\vfill % Fill the rest of the page with whitespace

\end{titlepage}



\section*{Введение}\par
В стремительно развивающемся ландшафте цифровых финансов криптовалюты стали ключевой инновацией, бросившей вызов традиционным финансовым системам и представившей новую парадигму обмена ценностями. Классификация криптовалют - сложный и многогранный вопрос, имеющий решающее значение для понимания их влияния, регуляторных последствий и будущего развития. В данном обзоре литературы рассматриваются различные аспекты классификации криптовалюты, включая ее технологические основы, экономические характеристики и нормативно-правовую базу. Данный обзор был составлен на основе научных статей, отраслевых отчетов и нормативных руководств и призван обеспечить всестороннее понимание того, как классифицируются криптовалюты, и значения этих классификаций в глобальном контексте.

Осознание разницы природы криптовалют позволит понять примерный характер поведения их цены на рынке. Это позволит инвесторам строить приблизительные прогнозы изменения стоимости на этот актив и поможет дивесифицировать свои вложения, минимизировав риски больших потерь.

\section*{Методы}\par
Данный обзор является обобщением работ, опубликованных авторитетными источниками. Для поиска статей я использовала сервисы Google Scholar и JSTOR. Помимо этого, научные работы брались из научных журналов с высокими индексами цитируемости. 


\section*{Описание}\par
Цифровая валюта — это любая электронная валюта, которой можно рассчитываться и которую можно хранить онлайн. На российском рынке эмиссией цифрового рубля, то есть его выпуском, может заниматься только ЦБ. Криптовалюты же производятся в результате майнинга — сложных математических вычислений на устройствах, которые могут принадлежать кому угодно \cite{коречков2016экономическая}.

Начиная с 2009 года криптовалюты представляют собой новое явление на мировых финансовых рынках. Предоставляя альтернативные деньги и инвестиционные возможности, они функционируют вне централизованных финансовых институтов. Благодаря новой технологии, криптовалюты предоставляют менее дорогую альтернативу фиатным деньгам с точки зрения транзакционных издержек. Они также предоставляют возможность анонимных транзакций, что может быть полезно для тех, кто ценит конфиденциальность. Такая технология также обеспечивает высокий уровень безопасности, так как транзакции записываются и проверяются в децентрализованной сети.

Биткоин - первая криптовалюта, в которой использовалась успешно показавшая себя технология блокчейн\footnote{Блокче́йн (англ. blockchain, изначально block chain — цепь из блоков) — выстроенная по определённым правилам непрерывная последовательная цепочка блоков (связный список), содержащих информацию. Связь между блоками обеспечивается не только нумерацией, но и тем, что каждый блок содержит свою собственную хеш-сумму и хеш-сумму предыдущего блока}
Усовершенствование технологии Биткоина привело к созданию новых цифровых валют, объединенных в группу под названием "альткоины"\ , куда вошли все цифровые валюты кроме Биткоина \cite{ciaian2018virtual}. Альткоины составляют примерно 50\% от рыночных объемов торгов и состоят более чем из 13 тысяч различных криптовалют.

На рис.1 представлены криптовалюты с наибольшей рыночной капитализацией на 10.12.23.
(данные взяты с сайта \cite{coinmarketcup})

\begin{figure}[!tbp]
  \centering
  \subfloat[Источник: coinmarketcup.com на 10 декабря 2023]
{\includegraphics[width=0.5\textwidth]{images/diagram.jpg}\label{fig:f1}}
  \hfill
{\includegraphics[width=0.4\textwidth]{images/altcoin.jpg}\label{fig:f2}}
  \caption{}
\end{figure}


В современном мире криптовалюты не только являются платежным средством, но и включают в себя различные финансовые инфраструктуры и инструменты, построенные на платформе смарт-контрактов\footnote{Смарт-контракт (от англ. smart contract) - программа, которая отслеживает и обеспечивает исполнение обязательств при сделках}, которые выпускают цифровые валюты с различными функциями. 

Согласно исследованиям \cite{mukhopadhyay2018ethereum} и \cite{victor2019measuring}, криптоактивы можно разделить на 2 группы: монеты и токены, - в зависимости от того, чью технологию шифрования и передачи данных они используют. Монеты, например, Bitcoin, Ethereum и Litecoin, имеют свою собственную независимую сеть блокчейн и являются средствами платежа (то есть, используются для проведения расчетов между агентами рынка). 

Помимо обычных монет (коинов) авторы статьи \cite{ante2021influence} выделяют их подтип: стейблкоины. 
Их особенностью является тот факт, что они привязаны к какому-либо денежному активу.

В зависимости от актива, к которому привязан стейблкоин, авторы статьи \cite{qin2021cefi} разделяют их на централизованные и децентрализованные.

Все эти известные попытки создать вечную, стабильную валютно-финансовую систему были основаны на предпосылке централизованной структуры, где, например, правительство обеспечивает финансовую ценность валюты, с военной силой под его командованием.
Однако история показала, что валюта может также оцениваться используя вмененную стоимость, то есть предполагаемую стоимость, присвоенную валюте, которая может быть не связана с ее внутренней стоимостью и, например, может быть даже нулевой.

С появлением блокчейн и их децентрализованной, не требующей разрешения природы, возникли новые вмененные валюты. Одной из самых сильных инноваций блокчейна - передача и торговля финансовых активов без доверенных посредников. Кроме того, децентрализованные финансы (DeFi), новая подобласть блокчейна, специализируется на развитии финансовых технологий и услуг поверх смарт-контрактов. DeFi поддерживает большинство продуктов, доступных в CeFi: биржи активов, кредиты, торговля с кредитным плечом, децентрализованное голосование за управление, стабильные монеты. Ассортимент продуктов быстро расширяется, и некоторые из более сложных продуктов, такие как опционы и деривативы, быстро развиваются.


В отличие от традиционных централизованных финансов DeFi\footnote{DeFi (Decentralized Finance) - модель организации финансов, основанная на оказании услуг без участия финансовых посредников или централизованных процессов} предлагает три отличительные особенности:

1. Прозрачность. В DeFi пользователь может ознакомиться с точными правилами, по которым работают финансовые активы и продукты. DeFi пытается избежать частных соглашений, обратных сделок и централизации, которые являются 
существенными ограничивающими факторами прозрачности CeFi.

2. Контроль. DeFi предлагает пользователям контроль, позволяя им оставаться хранителями своих активов, т.е. никто не должен иметь возможности цензурировать, перемещать или уничтожать активы пользователей без их согласия. 

3. Доступность. Любой человек, имеющий средний компьютер, подключение к Интернету ноу-хау может создавать и внедрять продукты DeFi, в то время как блокчейн и его распределенная сеть майнеров эффективно управляют приложением DeFi. Более того финансовая выгода DeFi также представляет собой значительный контраст по сравнению с CeFi.

В данной статье дерево решений CeFi - DeFi: В связи с отсутствием определения, когда речь идет о DeFi, в статье подготовлено  возможное дерево решений, которое может помочь классифицировать финансовый продукт или услугу как CeFi или DeFi. В этом дереве первым решающим вопросом является то, находятся ли финансовые активы находятся у пользователя, то есть сохраняет ли он контроль над своими активами. Если пользователь не контролирует активы и не сохраняет возможность совершать операции с активами без финансового посредника, то сервис является экземпляром CeFi. В противном случае мы задаемся вопросом, есть ли у кого-то возможность в одностороннем порядке цензурировать выполнение транзакции в одностороннем порядке. Такой мощный посредник указывает на существование посредника CeFi, в то время как расчеты по активам могут по-прежнему происходить децентрализованно, в соответствии с требованиями DeFi. Наконец, в статье задается вопрос, имеет ли субъект право единолично остановить или подвергнуть цензуре исполнение протокола. Если это так, то в статье утверждается, что протокол DeFi управляется централизованно. Если на этот последний вопрос можно ответить отрицательно, то рассматриваемый протокол будет квалифицироваться как чистый протокол DeFi. Авторы статьи являются первыми, кто с помощью трех простых и объективных вопросов, определяют является ли сервис экземпляром CeFi или DeFi. Их методология также подчеркивает, что граница между CeFi и DeFi не всегда так однозначна, как кажется на первый взгляд. 

Ниже визуализировано описанное дерево решений:

\includegraphics[width=0.85\textwidth]{images/defi.jpg}

Давайте вернемся к классификации криптовалют из статей \cite{mukhopadhyay2018ethereum},  \cite{victor2019measuring}, разделивших все активы на токены и монеты.

В отличие от монет токены создаются с помощью платформы смарт-контрактов, на уже существующих блокчейнах, используемыми другими цифровыми валютами. Например, токен ERC-20, может быть разработан только с помощью блокчейна Ethereum\cite{mukhopadhyay2018ethereum}. 

Авторы исследований \cite{sockin2023model}, \cite{howell2020initial} разделяют токены на utility (утилитарные, служебные) и security (инвестиционные или токены безопасности). Утилитарные токены предоставляют доступ к возможностям платформы. То есть, приобретая токен платформы, обладатель получает доступ к продукту проекта.  Например, токен LEO был создан для предоставления преимуществ трейдерам Bitfinex и позволял владельцу токена получать лучшие ставки и услуги при использовании Bitfinex. Токен безопасности позволяет владеть активами, а также предоставляет право получать выплату дивидендов и иметь право голоса \cite{dos2021bibliometric}. 

Авторы статьи \cite{ante2021influence} выделяют такой тип токенов, как NFT (non-fungible token) - это разновидность криптовалюты, которая создается с помощью
смарт-контрактов Ethereum. Особенностью такого токена является уникальность и взаимонезаменямость, что делает его подходящим для идентификации чего-либо уникальным способом. Использование NFT
в смарт-контрактах позволяет создателю легко доказать существование
и право собственности на цифровые активы в виде изображений и произведений искусства, билетов на мероприятия и т. д.
 Более того, создатель также может получать роялти каждый раз при
успешной торговли на любом рынке NFT или при Р2Р обмене. 
Возможность торговли, высокая ликвидность и удобное взаимодействие позволяют NFT стать перспективным решением для защиты интеллектуальной собственности (ИС). Уникальность кода каждого токена придает каждому токену особую ценность.

Углубившись в эту тему, можно разделить NFT на несколько групп в зависимости от протокола на котором работает каждый токен. Авторы статьи \cite{ante2022non} подробно рассказывают о том, как криптовалютные протоколы влияют взаимозаменяемость токенов.

Наиболее распространенным стандартом токенов является ERC-20. Он вводит концепцию взаимозаменяемых токенов, которые могут быть выпущены поверх Ethereum после выполнения
требованиям. Согласно этому стандарту, токены не отличаются друг от друга (как по типу, так и по стоимости). Произвольный токен всегда равен всем другим
токенам. Это ежегожно стимулирует ажиотаж вокруг первичных предложений монет (ICO).

В отличие от этого, тип токенов ERC-721 уникален и может быть отличим от других токенов. В частности, каждый 
NFT имеет уникальную пару, состоящую из номера и адреса контракта. 

Другой стандарт ERC-1155 (Multi Token Standard) может учитывать в не только номер, контракта, но и другую
настраиваемую информацию, такую как метаданные, время блокировки, дату, поставку, и т.д.

Приведем иллюстрацию, чтобы показать
 вышеупомянутые различия:
 \vspace{0.5em}

\includegraphics[width=0.85\textwidth]{images/protocols.jpg}

\section*{Анализ}\par
В настоящее время нет четкого и однозначного различия между CeFi и DeFi, поэтому приведенное в статье\cite{mukhopadhyay2018ethereum} дерево решений, позволяющее классифицировать активы в качестве  CeFi и DeFi выглядит очень актуальным.

Также обзоры литературы, проведенные в статьях \cite{turcomat} и \cite{wu2018classification} охватывают большое количество источников и позволяют всесторонне изучить тему классификации криптовалют не только с точки зрения их технической составляющей, но и с юридической.

Наглядная визуализация и подробное объяснение схемы работы различных протоколов в статье \cite{wu2018classification} существенно упрощает восприятие этой непростой темы и позволяет понять техническую часть работы протоколов NFT.

\printbibliography

\end{document}
